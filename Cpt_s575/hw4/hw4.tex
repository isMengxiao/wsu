\documentclass[]{article}
\usepackage{lmodern}
\usepackage{amssymb,amsmath}
\usepackage{ifxetex,ifluatex}
\usepackage{fixltx2e} % provides \textsubscript
\ifnum 0\ifxetex 1\fi\ifluatex 1\fi=0 % if pdftex
  \usepackage[T1]{fontenc}
  \usepackage[utf8]{inputenc}
\else % if luatex or xelatex
  \ifxetex
    \usepackage{mathspec}
  \else
    \usepackage{fontspec}
  \fi
  \defaultfontfeatures{Ligatures=TeX,Scale=MatchLowercase}
\fi
% use upquote if available, for straight quotes in verbatim environments
\IfFileExists{upquote.sty}{\usepackage{upquote}}{}
% use microtype if available
\IfFileExists{microtype.sty}{%
\usepackage{microtype}
\UseMicrotypeSet[protrusion]{basicmath} % disable protrusion for tt fonts
}{}
\usepackage[margin=1in]{geometry}
\usepackage{hyperref}
\hypersetup{unicode=true,
            pdftitle={Cpts575 Hw4},
            pdfauthor={Mengxiao},
            pdfborder={0 0 0},
            breaklinks=true}
\urlstyle{same}  % don't use monospace font for urls
\usepackage{color}
\usepackage{fancyvrb}
\newcommand{\VerbBar}{|}
\newcommand{\VERB}{\Verb[commandchars=\\\{\}]}
\DefineVerbatimEnvironment{Highlighting}{Verbatim}{commandchars=\\\{\}}
% Add ',fontsize=\small' for more characters per line
\usepackage{framed}
\definecolor{shadecolor}{RGB}{248,248,248}
\newenvironment{Shaded}{\begin{snugshade}}{\end{snugshade}}
\newcommand{\AlertTok}[1]{\textcolor[rgb]{0.94,0.16,0.16}{#1}}
\newcommand{\AnnotationTok}[1]{\textcolor[rgb]{0.56,0.35,0.01}{\textbf{\textit{#1}}}}
\newcommand{\AttributeTok}[1]{\textcolor[rgb]{0.77,0.63,0.00}{#1}}
\newcommand{\BaseNTok}[1]{\textcolor[rgb]{0.00,0.00,0.81}{#1}}
\newcommand{\BuiltInTok}[1]{#1}
\newcommand{\CharTok}[1]{\textcolor[rgb]{0.31,0.60,0.02}{#1}}
\newcommand{\CommentTok}[1]{\textcolor[rgb]{0.56,0.35,0.01}{\textit{#1}}}
\newcommand{\CommentVarTok}[1]{\textcolor[rgb]{0.56,0.35,0.01}{\textbf{\textit{#1}}}}
\newcommand{\ConstantTok}[1]{\textcolor[rgb]{0.00,0.00,0.00}{#1}}
\newcommand{\ControlFlowTok}[1]{\textcolor[rgb]{0.13,0.29,0.53}{\textbf{#1}}}
\newcommand{\DataTypeTok}[1]{\textcolor[rgb]{0.13,0.29,0.53}{#1}}
\newcommand{\DecValTok}[1]{\textcolor[rgb]{0.00,0.00,0.81}{#1}}
\newcommand{\DocumentationTok}[1]{\textcolor[rgb]{0.56,0.35,0.01}{\textbf{\textit{#1}}}}
\newcommand{\ErrorTok}[1]{\textcolor[rgb]{0.64,0.00,0.00}{\textbf{#1}}}
\newcommand{\ExtensionTok}[1]{#1}
\newcommand{\FloatTok}[1]{\textcolor[rgb]{0.00,0.00,0.81}{#1}}
\newcommand{\FunctionTok}[1]{\textcolor[rgb]{0.00,0.00,0.00}{#1}}
\newcommand{\ImportTok}[1]{#1}
\newcommand{\InformationTok}[1]{\textcolor[rgb]{0.56,0.35,0.01}{\textbf{\textit{#1}}}}
\newcommand{\KeywordTok}[1]{\textcolor[rgb]{0.13,0.29,0.53}{\textbf{#1}}}
\newcommand{\NormalTok}[1]{#1}
\newcommand{\OperatorTok}[1]{\textcolor[rgb]{0.81,0.36,0.00}{\textbf{#1}}}
\newcommand{\OtherTok}[1]{\textcolor[rgb]{0.56,0.35,0.01}{#1}}
\newcommand{\PreprocessorTok}[1]{\textcolor[rgb]{0.56,0.35,0.01}{\textit{#1}}}
\newcommand{\RegionMarkerTok}[1]{#1}
\newcommand{\SpecialCharTok}[1]{\textcolor[rgb]{0.00,0.00,0.00}{#1}}
\newcommand{\SpecialStringTok}[1]{\textcolor[rgb]{0.31,0.60,0.02}{#1}}
\newcommand{\StringTok}[1]{\textcolor[rgb]{0.31,0.60,0.02}{#1}}
\newcommand{\VariableTok}[1]{\textcolor[rgb]{0.00,0.00,0.00}{#1}}
\newcommand{\VerbatimStringTok}[1]{\textcolor[rgb]{0.31,0.60,0.02}{#1}}
\newcommand{\WarningTok}[1]{\textcolor[rgb]{0.56,0.35,0.01}{\textbf{\textit{#1}}}}
\usepackage{graphicx,grffile}
\makeatletter
\def\maxwidth{\ifdim\Gin@nat@width>\linewidth\linewidth\else\Gin@nat@width\fi}
\def\maxheight{\ifdim\Gin@nat@height>\textheight\textheight\else\Gin@nat@height\fi}
\makeatother
% Scale images if necessary, so that they will not overflow the page
% margins by default, and it is still possible to overwrite the defaults
% using explicit options in \includegraphics[width, height, ...]{}
\setkeys{Gin}{width=\maxwidth,height=\maxheight,keepaspectratio}
\IfFileExists{parskip.sty}{%
\usepackage{parskip}
}{% else
\setlength{\parindent}{0pt}
\setlength{\parskip}{6pt plus 2pt minus 1pt}
}
\setlength{\emergencystretch}{3em}  % prevent overfull lines
\providecommand{\tightlist}{%
  \setlength{\itemsep}{0pt}\setlength{\parskip}{0pt}}
\setcounter{secnumdepth}{0}
% Redefines (sub)paragraphs to behave more like sections
\ifx\paragraph\undefined\else
\let\oldparagraph\paragraph
\renewcommand{\paragraph}[1]{\oldparagraph{#1}\mbox{}}
\fi
\ifx\subparagraph\undefined\else
\let\oldsubparagraph\subparagraph
\renewcommand{\subparagraph}[1]{\oldsubparagraph{#1}\mbox{}}
\fi

%%% Use protect on footnotes to avoid problems with footnotes in titles
\let\rmarkdownfootnote\footnote%
\def\footnote{\protect\rmarkdownfootnote}

%%% Change title format to be more compact
\usepackage{titling}

% Create subtitle command for use in maketitle
\providecommand{\subtitle}[1]{
  \posttitle{
    \begin{center}\large#1\end{center}
    }
}

\setlength{\droptitle}{-2em}

  \title{Cpts575 Hw4}
    \pretitle{\vspace{\droptitle}\centering\huge}
  \posttitle{\par}
    \author{Mengxiao}
    \preauthor{\centering\large\emph}
  \postauthor{\par}
    \date{}
    \predate{}\postdate{}
  

\begin{document}
\maketitle

\hypertarget{part-1}{%
\section{Part 1}\label{part-1}}

\begin{Shaded}
\begin{Highlighting}[]
\KeywordTok{library}\NormalTok{(dplyr)}
\KeywordTok{library}\NormalTok{(graphics)}
\NormalTok{Auto =}\StringTok{ }\KeywordTok{read.csv}\NormalTok{(}\StringTok{"https://scads.eecs.wsu.edu/wp-content/uploads/2017/09/Auto.csv"}\NormalTok{, }\DataTypeTok{na.string =} \StringTok{'?'}\NormalTok{)}
\NormalTok{Auto =}\StringTok{ }\KeywordTok{na.omit}\NormalTok{(Auto)}
\CommentTok{#Auto = Auto[Auto$horsepower != '?',] #Moving out the missing data}
\end{Highlighting}
\end{Shaded}

\hypertarget{a.-produce-a-scatterplot-matrix}{%
\subsection{a. Produce a scatterplot
matrix}\label{a.-produce-a-scatterplot-matrix}}

\begin{Shaded}
\begin{Highlighting}[]
\KeywordTok{pairs}\NormalTok{(Auto)}
\end{Highlighting}
\end{Shaded}

\includegraphics{hw4_files/figure-latex/unnamed-chunk-2-1.pdf}

\hypertarget{b.-compute-the-matrix-of-correlations.}{%
\subsection{b. Compute the matrix of
correlations.}\label{b.-compute-the-matrix-of-correlations.}}

\begin{Shaded}
\begin{Highlighting}[]
\NormalTok{Auto2 =}\StringTok{ }\NormalTok{Auto }\OperatorTok\StringTok{ }\NormalTok{dplyr}\OperatorTok{::}\KeywordTok{select}\NormalTok{(}\OperatorTok{-}\NormalTok{name)}
\KeywordTok{cor}\NormalTok{(Auto2)}
\end{Highlighting}
\end{Shaded}

\begin{verbatim}
##                     mpg  cylinders displacement horsepower     weight
## mpg           1.0000000 -0.7776175   -0.8051269 -0.7784268 -0.8322442
## cylinders    -0.7776175  1.0000000    0.9508233  0.8429834  0.8975273
## displacement -0.8051269  0.9508233    1.0000000  0.8972570  0.9329944
## horsepower   -0.7784268  0.8429834    0.8972570  1.0000000  0.8645377
## weight       -0.8322442  0.8975273    0.9329944  0.8645377  1.0000000
## acceleration  0.4233285 -0.5046834   -0.5438005 -0.6891955 -0.4168392
## year          0.5805410 -0.3456474   -0.3698552 -0.4163615 -0.3091199
## origin        0.5652088 -0.5689316   -0.6145351 -0.4551715 -0.5850054
##              acceleration       year     origin
## mpg             0.4233285  0.5805410  0.5652088
## cylinders      -0.5046834 -0.3456474 -0.5689316
## displacement   -0.5438005 -0.3698552 -0.6145351
## horsepower     -0.6891955 -0.4163615 -0.4551715
## weight         -0.4168392 -0.3091199 -0.5850054
## acceleration    1.0000000  0.2903161  0.2127458
## year            0.2903161  1.0000000  0.1815277
## origin          0.2127458  0.1815277  1.0000000
\end{verbatim}

\hypertarget{c.-perform-a-multiple-linear-regression.}{%
\subsection{c. Perform a multiple linear
regression.}\label{c.-perform-a-multiple-linear-regression.}}

\begin{Shaded}
\begin{Highlighting}[]
\NormalTok{lr =}\StringTok{ }\KeywordTok{lm}\NormalTok{(mpg}\OperatorTok{~}\NormalTok{., }\DataTypeTok{data =}\NormalTok{ Auto2)}
\KeywordTok{summary}\NormalTok{(lr)}
\end{Highlighting}
\end{Shaded}

\begin{verbatim}
## 
## Call:
## lm(formula = mpg ~ ., data = Auto2)
## 
## Residuals:
##     Min      1Q  Median      3Q     Max 
## -9.5903 -2.1565 -0.1169  1.8690 13.0604 
## 
## Coefficients:
##                Estimate Std. Error t value Pr(>|t|)    
## (Intercept)  -17.218435   4.644294  -3.707  0.00024 ***
## cylinders     -0.493376   0.323282  -1.526  0.12780    
## displacement   0.019896   0.007515   2.647  0.00844 ** 
## horsepower    -0.016951   0.013787  -1.230  0.21963    
## weight        -0.006474   0.000652  -9.929  < 2e-16 ***
## acceleration   0.080576   0.098845   0.815  0.41548    
## year           0.750773   0.050973  14.729  < 2e-16 ***
## origin         1.426141   0.278136   5.127 4.67e-07 ***
## ---
## Signif. codes:  0 '***' 0.001 '**' 0.01 '*' 0.05 '.' 0.1 ' ' 1
## 
## Residual standard error: 3.328 on 384 degrees of freedom
## Multiple R-squared:  0.8215, Adjusted R-squared:  0.8182 
## F-statistic: 252.4 on 7 and 384 DF,  p-value: < 2.2e-16
\end{verbatim}

\hypertarget{i.-i-think-the-displacement-weight-year-and-origin-have-the-statistically-significant-relationship-with-the-mpg-since-their-p-value-are-less-than-0.05-they-are-significant.}{%
\subsubsection{i. I think the `displacement', `weight', `year' and
`origin' have the statistically significant relationship with the `mpg',
since their P-value are less than `0.05', they are
significant.}\label{i.-i-think-the-displacement-weight-year-and-origin-have-the-statistically-significant-relationship-with-the-mpg-since-their-p-value-are-less-than-0.05-they-are-significant.}}

\hypertarget{ii.-means-when-the-value-of-displacement-increase-1-the-mpg-will-increase-0.019896.}{%
\subsubsection{ii. Means when the value of displacement increase 1\%,
the mpg will increase
0.019896\%.}\label{ii.-means-when-the-value-of-displacement-increase-1-the-mpg-will-increase-0.019896.}}

\hypertarget{d.-produce-diagnostic-plots-of-the-linear-regression-fit.}{%
\subsection{d. Produce diagnostic plots of the linear regression
fit.}\label{d.-produce-diagnostic-plots-of-the-linear-regression-fit.}}

\begin{Shaded}
\begin{Highlighting}[]
\KeywordTok{plot}\NormalTok{(lr)}
\end{Highlighting}
\end{Shaded}

\includegraphics{hw4_files/figure-latex/unnamed-chunk-5-1.pdf}
\includegraphics{hw4_files/figure-latex/unnamed-chunk-5-2.pdf}
\includegraphics{hw4_files/figure-latex/unnamed-chunk-5-3.pdf}
\includegraphics{hw4_files/figure-latex/unnamed-chunk-5-4.pdf}

\hypertarget{the-residual-plots-looks-good-but-still-have-some-outliers.}{%
\subsubsection{The residual plots looks good, but still have some
outliers.}\label{the-residual-plots-looks-good-but-still-have-some-outliers.}}

\hypertarget{yes-it-identifies-some-unusually-outliers}{%
\subsubsection{Yes, it identifies some unusually
outliers}\label{yes-it-identifies-some-unusually-outliers}}

\hypertarget{e.-fit-linear-regression-models-with-interaction-effects.}{%
\subsection{e. Fit linear regression models with interaction
effects.}\label{e.-fit-linear-regression-models-with-interaction-effects.}}

\begin{Shaded}
\begin{Highlighting}[]
\NormalTok{lm_e =}\StringTok{ }\KeywordTok{lm}\NormalTok{(mpg}\OperatorTok{~}\NormalTok{cylinders}\OperatorTok{*}\NormalTok{displacement }\OperatorTok{+}\StringTok{ }\NormalTok{weight}\OperatorTok{*}\NormalTok{displacement, }\DataTypeTok{data=}\NormalTok{Auto2)}
\KeywordTok{summary}\NormalTok{(lm_e)}
\end{Highlighting}
\end{Shaded}

\begin{verbatim}
## 
## Call:
## lm(formula = mpg ~ cylinders * displacement + weight * displacement, 
##     data = Auto2)
## 
## Residuals:
##      Min       1Q   Median       3Q      Max 
## -13.2934  -2.5184  -0.3476   1.8399  17.7723 
## 
## Coefficients:
##                          Estimate Std. Error t value Pr(>|t|)    
## (Intercept)             5.262e+01  2.237e+00  23.519  < 2e-16 ***
## cylinders               7.606e-01  7.669e-01   0.992    0.322    
## displacement           -7.351e-02  1.669e-02  -4.403 1.38e-05 ***
## weight                 -9.888e-03  1.329e-03  -7.438 6.69e-13 ***
## cylinders:displacement -2.986e-03  3.426e-03  -0.872    0.384    
## displacement:weight     2.128e-05  5.002e-06   4.254 2.64e-05 ***
## ---
## Signif. codes:  0 '***' 0.001 '**' 0.01 '*' 0.05 '.' 0.1 ' ' 1
## 
## Residual standard error: 4.103 on 386 degrees of freedom
## Multiple R-squared:  0.7272, Adjusted R-squared:  0.7237 
## F-statistic: 205.8 on 5 and 386 DF,  p-value: < 2.2e-16
\end{verbatim}

We can see from the summay that displacement and weight have
statistically signifcant relationship, but the relationship between
cylinders and displacement is not significant.

\hypertarget{f.-try-transformations-of-the-variables-with-x3-and-logx.}{%
\subsection{f. Try transformations of the variables with X\^{}3 and
log(X).}\label{f.-try-transformations-of-the-variables-with-x3-and-logx.}}

\begin{Shaded}
\begin{Highlighting}[]
\KeywordTok{plot}\NormalTok{((Auto2}\OperatorTok{$}\NormalTok{displacement)}\OperatorTok{^}\DecValTok{3}\NormalTok{, Auto2}\OperatorTok{$}\NormalTok{mpg)}
\end{Highlighting}
\end{Shaded}

\includegraphics{hw4_files/figure-latex/unnamed-chunk-7-1.pdf}

\begin{Shaded}
\begin{Highlighting}[]
\KeywordTok{plot}\NormalTok{(}\KeywordTok{sqrt}\NormalTok{(Auto2}\OperatorTok{$}\NormalTok{displacement), Auto2}\OperatorTok{$}\NormalTok{mpg)}
\end{Highlighting}
\end{Shaded}

\includegraphics{hw4_files/figure-latex/unnamed-chunk-7-2.pdf}

\hypertarget{it-looks-like-the-distribution-is-more-aggregated-of-x3.}{%
\subsubsection{It looks like the distribution is more aggregated of
X\^{}3.}\label{it-looks-like-the-distribution-is-more-aggregated-of-x3.}}

\hypertarget{part-2}{%
\section{Part 2}\label{part-2}}

\begin{Shaded}
\begin{Highlighting}[]
\KeywordTok{library}\NormalTok{(MASS)}
\KeywordTok{head}\NormalTok{(Boston)}
\end{Highlighting}
\end{Shaded}

\begin{verbatim}
##      crim zn indus chas   nox    rm  age    dis rad tax ptratio  black
## 1 0.00632 18  2.31    0 0.538 6.575 65.2 4.0900   1 296    15.3 396.90
## 2 0.02731  0  7.07    0 0.469 6.421 78.9 4.9671   2 242    17.8 396.90
## 3 0.02729  0  7.07    0 0.469 7.185 61.1 4.9671   2 242    17.8 392.83
## 4 0.03237  0  2.18    0 0.458 6.998 45.8 6.0622   3 222    18.7 394.63
## 5 0.06905  0  2.18    0 0.458 7.147 54.2 6.0622   3 222    18.7 396.90
## 6 0.02985  0  2.18    0 0.458 6.430 58.7 6.0622   3 222    18.7 394.12
##   lstat medv
## 1  4.98 24.0
## 2  9.14 21.6
## 3  4.03 34.7
## 4  2.94 33.4
## 5  5.33 36.2
## 6  5.21 28.7
\end{verbatim}

\hypertarget{a.}{%
\subsection{a.}\label{a.}}

\begin{Shaded}
\begin{Highlighting}[]
\NormalTok{lm_zn =}\StringTok{ }\KeywordTok{lm}\NormalTok{(crim}\OperatorTok{~}\NormalTok{zn, }\DataTypeTok{data=}\NormalTok{Boston)}
\NormalTok{lm_indus =}\StringTok{ }\KeywordTok{lm}\NormalTok{(crim}\OperatorTok{~}\NormalTok{indus, }\DataTypeTok{data=}\NormalTok{Boston)}
\NormalTok{lm_chas =}\StringTok{ }\KeywordTok{lm}\NormalTok{(crim}\OperatorTok{~}\NormalTok{chas, }\DataTypeTok{data=}\NormalTok{Boston)}
\NormalTok{lm_nox =}\StringTok{ }\KeywordTok{lm}\NormalTok{(crim}\OperatorTok{~}\NormalTok{nox, }\DataTypeTok{data=}\NormalTok{Boston)}
\NormalTok{lm_rm =}\StringTok{ }\KeywordTok{lm}\NormalTok{(crim}\OperatorTok{~}\NormalTok{rm, }\DataTypeTok{data=}\NormalTok{Boston)}
\NormalTok{lm_age =}\StringTok{ }\KeywordTok{lm}\NormalTok{(crim}\OperatorTok{~}\NormalTok{age, }\DataTypeTok{data=}\NormalTok{Boston)}
\NormalTok{lm_dis =}\StringTok{ }\KeywordTok{lm}\NormalTok{(crim}\OperatorTok{~}\NormalTok{dis, }\DataTypeTok{data=}\NormalTok{Boston)}
\NormalTok{lm_rad =}\StringTok{ }\KeywordTok{lm}\NormalTok{(crim}\OperatorTok{~}\NormalTok{rad, }\DataTypeTok{data=}\NormalTok{Boston)}
\NormalTok{lm_tax =}\StringTok{ }\KeywordTok{lm}\NormalTok{(crim}\OperatorTok{~}\NormalTok{tax, }\DataTypeTok{data=}\NormalTok{Boston)}
\NormalTok{lm_ptratio =}\StringTok{ }\KeywordTok{lm}\NormalTok{(crim}\OperatorTok{~}\NormalTok{ptratio, }\DataTypeTok{data=}\NormalTok{Boston)}
\NormalTok{lm_black =}\StringTok{ }\KeywordTok{lm}\NormalTok{(crim}\OperatorTok{~}\NormalTok{black, }\DataTypeTok{data=}\NormalTok{Boston)}
\NormalTok{lm_lstat =}\StringTok{ }\KeywordTok{lm}\NormalTok{(crim}\OperatorTok{~}\NormalTok{lstat, }\DataTypeTok{data=}\NormalTok{Boston)}
\NormalTok{lm_medv =}\StringTok{ }\KeywordTok{lm}\NormalTok{(crim}\OperatorTok{~}\NormalTok{medv, }\DataTypeTok{data=}\NormalTok{Boston)}
\end{Highlighting}
\end{Shaded}

I find that only `chas' don't have statistically significant
relationship with crim, all of other variables have significant
relationship.

\hypertarget{b.}{%
\subsection{b.}\label{b.}}

\begin{Shaded}
\begin{Highlighting}[]
\NormalTok{lm_mul =}\StringTok{ }\KeywordTok{lm}\NormalTok{(crim}\OperatorTok{~}\NormalTok{., }\DataTypeTok{data=}\NormalTok{Boston)}
\KeywordTok{summary}\NormalTok{(lm_mul)}
\end{Highlighting}
\end{Shaded}

\begin{verbatim}
## 
## Call:
## lm(formula = crim ~ ., data = Boston)
## 
## Residuals:
##    Min     1Q Median     3Q    Max 
## -9.924 -2.120 -0.353  1.019 75.051 
## 
## Coefficients:
##               Estimate Std. Error t value Pr(>|t|)    
## (Intercept)  17.033228   7.234903   2.354 0.018949 *  
## zn            0.044855   0.018734   2.394 0.017025 *  
## indus        -0.063855   0.083407  -0.766 0.444294    
## chas         -0.749134   1.180147  -0.635 0.525867    
## nox         -10.313535   5.275536  -1.955 0.051152 .  
## rm            0.430131   0.612830   0.702 0.483089    
## age           0.001452   0.017925   0.081 0.935488    
## dis          -0.987176   0.281817  -3.503 0.000502 ***
## rad           0.588209   0.088049   6.680 6.46e-11 ***
## tax          -0.003780   0.005156  -0.733 0.463793    
## ptratio      -0.271081   0.186450  -1.454 0.146611    
## black        -0.007538   0.003673  -2.052 0.040702 *  
## lstat         0.126211   0.075725   1.667 0.096208 .  
## medv         -0.198887   0.060516  -3.287 0.001087 ** 
## ---
## Signif. codes:  0 '***' 0.001 '**' 0.01 '*' 0.05 '.' 0.1 ' ' 1
## 
## Residual standard error: 6.439 on 492 degrees of freedom
## Multiple R-squared:  0.454,  Adjusted R-squared:  0.4396 
## F-statistic: 31.47 on 13 and 492 DF,  p-value: < 2.2e-16
\end{verbatim}

In my opinion, we can reject the `zn', `dis', `rad', `black' and `medv',
since their P-value are all less than 0.05.

\hypertarget{c.how-do-your-results-from-a-compare-to-your-results-from-b}{%
\subsection{c.How do your results from (a) compare to your results from
(b)?}\label{c.how-do-your-results-from-a-compare-to-your-results-from-b}}

\begin{Shaded}
\begin{Highlighting}[]
\NormalTok{simple =}\StringTok{ }\KeywordTok{c}\NormalTok{(lm_zn}\OperatorTok{$}\NormalTok{coefficients[}\DecValTok{2}\NormalTok{], lm_indus}\OperatorTok{$}\NormalTok{coefficients[}\DecValTok{2}\NormalTok{], lm_chas}\OperatorTok{$}\NormalTok{coefficients[}\DecValTok{2}\NormalTok{], lm_nox}\OperatorTok{$}\NormalTok{coefficients[}\DecValTok{2}\NormalTok{],}
\NormalTok{           lm_rm}\OperatorTok{$}\NormalTok{coefficients[}\DecValTok{2}\NormalTok{], lm_age}\OperatorTok{$}\NormalTok{coefficients[}\DecValTok{2}\NormalTok{], lm_dis}\OperatorTok{$}\NormalTok{coefficients[}\DecValTok{2}\NormalTok{], lm_rad}\OperatorTok{$}\NormalTok{coefficients[}\DecValTok{2}\NormalTok{],}
\NormalTok{           lm_tax}\OperatorTok{$}\NormalTok{coefficients[}\DecValTok{2}\NormalTok{], lm_ptratio}\OperatorTok{$}\NormalTok{coefficients[}\DecValTok{2}\NormalTok{], lm_black}\OperatorTok{$}\NormalTok{coefficients[}\DecValTok{2}\NormalTok{], lm_lstat}\OperatorTok{$}\NormalTok{coefficients[}\DecValTok{2}\NormalTok{],}
\NormalTok{           lm_medv}\OperatorTok{$}\NormalTok{coefficients[}\DecValTok{2}\NormalTok{])}
\NormalTok{multi =}\StringTok{ }\KeywordTok{c}\NormalTok{(lm_mul}\OperatorTok{$}\NormalTok{coefficients)}
\NormalTok{multi =}\StringTok{ }\NormalTok{multi[}\OperatorTok{-}\DecValTok{1}\NormalTok{]}
\KeywordTok{plot}\NormalTok{(simple, multi)}
\end{Highlighting}
\end{Shaded}

\includegraphics{hw4_files/figure-latex/unnamed-chunk-11-1.pdf} The
coefficients of simple is much higher than it of multiple, that the
arrange of simple is 0 to 30 and for multiple is from -10 to 0. In my
opinion, it is because simple predict only shows whether two variables
have relationship and the rate of relation, but the multiple predict
shows the rate of different variables' influence.

\hypertarget{d.-is-there-evidence-of-non-linear-association-between-any-of-the-predictors-and-the-response}{%
\subsection{d. Is there evidence of non-linear association between any
of the predictors and the
response?}\label{d.-is-there-evidence-of-non-linear-association-between-any-of-the-predictors-and-the-response}}

\begin{Shaded}
\begin{Highlighting}[]
\NormalTok{lm_zn2 =}\StringTok{ }\KeywordTok{lm}\NormalTok{(crim}\OperatorTok{~}\KeywordTok{poly}\NormalTok{(zn, }\DecValTok{3}\NormalTok{), }\DataTypeTok{data=}\NormalTok{Boston)}
\NormalTok{lm_indus2 =}\StringTok{ }\KeywordTok{lm}\NormalTok{(crim}\OperatorTok{~}\KeywordTok{poly}\NormalTok{(indus, }\DecValTok{3}\NormalTok{), }\DataTypeTok{data=}\NormalTok{Boston)}
\NormalTok{lm_nox2 =}\StringTok{ }\KeywordTok{lm}\NormalTok{(crim}\OperatorTok{~}\KeywordTok{poly}\NormalTok{(nox, }\DecValTok{3}\NormalTok{), }\DataTypeTok{data=}\NormalTok{Boston)}
\NormalTok{lm_rm2 =}\StringTok{ }\KeywordTok{lm}\NormalTok{(crim}\OperatorTok{~}\KeywordTok{poly}\NormalTok{(rm, }\DecValTok{3}\NormalTok{), }\DataTypeTok{data=}\NormalTok{Boston)}
\NormalTok{lm_age2 =}\StringTok{ }\KeywordTok{lm}\NormalTok{(crim}\OperatorTok{~}\KeywordTok{poly}\NormalTok{(age, }\DecValTok{3}\NormalTok{), }\DataTypeTok{data=}\NormalTok{Boston)}
\NormalTok{lm_dis2 =}\StringTok{ }\KeywordTok{lm}\NormalTok{(crim}\OperatorTok{~}\KeywordTok{poly}\NormalTok{(dis, }\DecValTok{3}\NormalTok{), }\DataTypeTok{data=}\NormalTok{Boston)}
\NormalTok{lm_rad2 =}\StringTok{ }\KeywordTok{lm}\NormalTok{(crim}\OperatorTok{~}\KeywordTok{poly}\NormalTok{(rad, }\DecValTok{3}\NormalTok{), }\DataTypeTok{data=}\NormalTok{Boston)}
\NormalTok{lm_tax2 =}\StringTok{ }\KeywordTok{lm}\NormalTok{(crim}\OperatorTok{~}\KeywordTok{poly}\NormalTok{(tax, }\DecValTok{3}\NormalTok{), }\DataTypeTok{data=}\NormalTok{Boston)}
\NormalTok{lm_ptratio2 =}\StringTok{ }\KeywordTok{lm}\NormalTok{(crim}\OperatorTok{~}\KeywordTok{poly}\NormalTok{(ptratio, }\DecValTok{3}\NormalTok{), }\DataTypeTok{data=}\NormalTok{Boston)}
\NormalTok{lm_black2 =}\StringTok{ }\KeywordTok{lm}\NormalTok{(crim}\OperatorTok{~}\KeywordTok{poly}\NormalTok{(black, }\DecValTok{3}\NormalTok{), }\DataTypeTok{data=}\NormalTok{Boston)}
\NormalTok{lm_lstat2 =}\StringTok{ }\KeywordTok{lm}\NormalTok{(crim}\OperatorTok{~}\KeywordTok{poly}\NormalTok{(lstat, }\DecValTok{3}\NormalTok{), }\DataTypeTok{data=}\NormalTok{Boston)}
\NormalTok{lm_medv2 =}\StringTok{ }\KeywordTok{lm}\NormalTok{(crim}\OperatorTok{~}\KeywordTok{poly}\NormalTok{(medv, }\DecValTok{3}\NormalTok{), }\DataTypeTok{data=}\NormalTok{Boston)}

\KeywordTok{summary}\NormalTok{(lm_zn2)}
\KeywordTok{summary}\NormalTok{(lm_indus2)}
\KeywordTok{summary}\NormalTok{(lm_nox2)}
\KeywordTok{summary}\NormalTok{(lm_rm2)}
\KeywordTok{summary}\NormalTok{(lm_age2)}
\KeywordTok{summary}\NormalTok{(lm_dis2)}
\KeywordTok{summary}\NormalTok{(lm_rad2)}
\KeywordTok{summary}\NormalTok{(lm_tax2)}
\KeywordTok{summary}\NormalTok{(lm_ptratio2)}
\KeywordTok{summary}\NormalTok{(lm_black2)}
\KeywordTok{summary}\NormalTok{(lm_lstat2)}
\KeywordTok{summary}\NormalTok{(lm_medv2)}
\end{Highlighting}
\end{Shaded}

I have found that only the `black' don't have non-linear association,
since the P-value of quandratic and cubic coefficients are all higher
than 0.05. The other variables all have non-linear association, but some
of them only have quandratic association and the other have cubic.

\hypertarget{part-3.}{%
\section{Part 3.}\label{part-3.}}


\end{document}
